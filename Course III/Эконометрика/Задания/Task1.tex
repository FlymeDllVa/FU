%! Author = flyme
%! Date = 10.09.2020

% Preamble
\documentclass[11pt,a4paper]{article}

% Packages
\usepackage[utf8]{inputenc}
\usepackage[english, russian]{babel}
\usepackage{amsmath}
\usepackage{natbib}
\usepackage{graphicx}
\usepackage[T2A]{fontenc}
\usepackage{systeme}

% Document


\begin{titlepage}
  \title{Задача 1. Модель формирования национального доход}
  \author{Гриднев Дмитрий Владимирович }
  \date{2020}
\end{titlepage}

\begin{document}
  \begin{titlepage}
    \begin{center}
      \large

      \textbf{Федеральное государственное образовательное бюджетное учреждение высшего образования}
      \vspace{0.5cm}

      Финансовый университет при Правительстве Российской Федерации
      \vspace{0.25cm}

      Факультет информационных технологий и анализа больших данных

      Департамент анализа данных и машинного обучения
      \vfill

      Гриднев Дмитрий Владимирович, 3 курс, ПИ18-1
      \vfill

      \textsc{Работа по курсу}\\[5mm]

      {\LARGE Эконометрика}
    \bigskip

  \end{center}
  \vfill

  \newlength{\ML}
  \settowidth{\ML}{«\underline{\hspace{0.7cm}}» \underline{\hspace{2cm}}}

  \hfill\begin{minipage}{0.6\textwidth}
    Преподаватель\\
    Смирнова Елена Константиновна\\
  \end{minipage}%
  \vfill

    \vspace{2cm}
  \begin{center}
    Москва, 2020 г.
  \end{center}
  \end{titlepage}

  \section{Задача 1. Модель формирования национального дохода (Дж. М. Кейнс)}\label{sec:task1}

  Y, C -- зависимые переменные\\
  I -- независимые переменные\\

  \systeme{
    C_T=\small{a}Y_{T-1}+b+\varepsilon@{b>0,\quad 0<a<1}, Y_T=C_T+I_T+U\_T
  }
  \newline

  $Y_T, C_T$ -- зависимые переменные (y)\\
  $I_T$ -- независимые переменные (x)
  \newline

  Структурная форма в матричном виде: $$AY + BX =U$$

  \systeme{
    C_T-0*Y_T-a*Y_{T-1}-b+0*I_T=\varepsilon_T, -C_T+Y_T+0*Y_{T-1}-0*X_0-I=U_T
  }

  \begin{vmatrix}
    1 & 0\\
    1 & 1
  \end{vmatrix}*
  \begin{vmatrix}
    C_T\\
    Y_T
  \end{vmatrix}+
  \begin{vmatrix}
    -a & -b & 0\\
    0 & 0 & -1
  \end{vmatrix}*
  \begin{vmatrix}
    Y_{T-1}\\
    1\\
    1-
  \end{vmatrix}=
  \begin{vmatrix}
    \varepsilon_T\\
    U-
  \end{vmatrix}

  Приведенная форма:

  A^{-1}B=M\\
  Y=MX+A^{-1}U\\

  \begin{vmatrix}
    1 & 0\\
    1 & 1
  \end{vmatrix}*
  \begin{vmatrix}
    -a & -b & 0\\
    0 & 0 & -1
  \end{vmatrix}=
  \begin{vmatrix}
    a & b & 0\\
    a & b & 1
  \end{vmatrix}\\

  \begin{vmatrix}
    C_T\\
    Y_T
  \end{vmatrix}=
  \begin{vmatrix}
    a & b & 0\\
    a & b & 1
  \end{vmatrix}*
  \begin{vmatrix}
    Y_{T-1}\\
    1\\
    I_T
  \end{vmatrix}+
  \begin{vmatrix}
    1 & 0\\
    1 & 1
  \end{vmatrix}*
  \begin{vmatrix}
    \varepsilon_T\\
    U_T
  \end{vmatrix}

  \section{Задача 2. Макромодель Самуэльсона-Хикса (модель делового цикла экономики}\label{sec:task2}

  $Y_t, C_t, I_t, G_t$ -- зависимые переменные\\
  $Y_{t-1}, Y_{t-2}, G_{t-1}, x_0 $ -- независмые переменные\\

  \systeme{
    C_t=a*Y_{t-1}+b+\varepsilon_t@{0<a<1,b>0},
    I_t=c*(Y_{t-1}-Y_{t-2})+U_t@{c>0},
    G_t=d*G_{t-1}@{d>1},
    Y_t=C_t+I_t+G_t@{- $ тождество$}
  }\\

  Структурная форма в матричном виде:

  \begin{equation*}
   \begin{cases}
     C_t-0*I_t-0*G_T-0*Y_t-a*Y_{t-1}-0*Y_{t-2}-0*G_{t-1}-b=\varepsilon_t,
     \\
     0*C_t+I_t-0*G_t-0*Y_t-c*Y_{t-1}+c*Y_{t-2}-0*G_{t-1}-0=U_t,
     \\
     0*C_t-0*I_t+1*G_t-0*Y_t-0*Y_{t-2}-d*G_{t-1}-0=V_t,
     \\
     -1*C_t-1*I_t-1*G_t+1*Y_t-0*Y_{t-1}-0*Y_{t-2}-0*G_{t-1}-b=0
   \end{cases}
  \end{equation*}\\

  Приведенная форма:
  \newline\\
  \begin{vmatrix}
    1 & 0 & 0 & 0\\
    0 & 1 & 0 & 0\\
    0 & 0 & 1 & 0\\
    -1 & -1 & -1 & 1\\
  \end{vmatrix}*
  \begin{vmatrix}
    C_t\\
    I_t\\
    G_t\\
    Y_t
  \end{vmatrix}+
  \begin{vmatrix}
    -a & 0 & 0 & -b\\
    -c & c & 0 & 0\\
    0 & 0 & -d 0\\
    0 & 0 & 0 & 0
  \end{vmatrix}*
  \begin{vmatrix}
    Y_{t-1}\\
    Y_{t-2}\\
    G_{t-1}\\
    1
  \end{vmatrix}=
  \begin{vmatrix}
    \varepsilon_t\\
    U_t\\
    V_t\\
    0
  \end{vmatrix}
  \\

  Приведенная форма:
  \newline
  $$M=-A^{-1}*B$$
  $$Y=MX+A^{-1}*V$$

  M=-\begin{vmatrix}
    1 & 0 & 0 & 0\\
    0 & 1 & 0 & 0\\
    0 & 0 & 1 & 0\\
    1 & 1 & 1 & 1\\
  \end{vmatrix}*\begin{vmatrix}
                  -a & 0 & 0 & -b\\
                  -c & c & 0 & 0\\
                  0 & 0 & -d & 0\\
                  0 & 0 & 0 & 0
  \end{vmatrix}=\begin{vmatrix}
                  -a & 0 & 0 & -b\\
                  -c & c & 0 & 0\\
                  -a & 0 & -d & 0\\
                  -a-c & c & d & -b
  \end{vmatrix}=\begin{vmatrix}
                  a & 0 & 0 & b\\
                  c & c & 0 & 0\\
                  0 & 0 & d & 0\\
                  a+c & -c & d & b
  \end{vmatrix}\\

  \newline\\
  \begin{vmatrix}
    C_t\\
    I_t\\
    G_t\\
    Y_t
  \end{vmatrix}=
  \begin{vmatrix}
    a & 0 & 0 b\\
    c & -c & 0 & 0\\
    0 & 0 & d & 0\\
    a+c & -c & d & b
  \end{vmatrix}*
  \begin{vmatrix}
    Y_{t-1}\\
    Y_{t-2}\\
    G_{t-1}\\
    1
  \end{vmatrix}+
  \begin{vmatrix}
    1 & 0 & 0 & 0\\
    0 & 1 & 0 & 0\\
    0 & 0 & 1 & 0\\
    1 & 1 & 1 & 1
  \end{vmatrix}*
  \begin{vmatrix}
    \varepsilon_t\\
    U_t\\
    V_t\\
    0
  \end{vmatrix}

  \newline
  \section{Задача 3. Модель производственной функции Кобба-Дугласа}\label{sec:task3}

Труд (L) и капитал (K) служат основными факторами количества (Y) выпускаемой продукции.
Требуется составить спецификацию модели производственной функ-ции, которая даёт возможность объяснять величину выпуска продукции уровнем капитала и живого труда.\\
  Экономические законы:
  \begin{enumerate}
    \item Каждый из факторов производства необходим в том смысле, что если K = 0, или L = 0, то объем выпуска Y = 0;
    \item Уровень выпуска возрастает вместе с ростом каждого из факторов;
    \item Если один из факторов фиксирован, а другой возрастает, то каждая допол-нительная (предельная) единица возрастающего фактора менее полезна (в смысле прироста выпуска продукции), чем предыдущая единица (закон Госсена об убыва-нии предельной полезности);
    \item Имеет место постоянство отдачи от масштаба, то есть при увеличении каж-дого из факторов производства в $\mu$ раз объем выпуска тоже возрастает в $\mu$ раз.
  \end{enumerate}

  \newline
  $Y_t=(K_t^a-L_t^b)\cdotM+\varepsilon_t, \space\space
  \begin{vmatrix}
      a + b = 1\\
      0 \geq a \geq 1\\
      0 \geq b \geq 1
  \end{vmatrix}$

  \section{Задача 4. Модель корректиров¬ки размера дивидендов Линтнера}\label{sec:task4}

  \begin{equation*}
   \begin{cases}
     $D_t^* g * \Pi_t$ – тождество\\
     $D_t D_{t-1}=a(D_t^*-D_{t-1}^{\varepsilont}), a>0$
   \end{cases}
  \end{equation*}
  \begin{equation*}
   \begin{cases}
     $D_t^*=g\Pi_t$\\
     $D_t=a(g\dot\Pi_t-D_{t-1})+\varepsilon_t+D+{t-1}$
   \end{cases}
  \end{equation*}
  \begin{equation*}
   \begin{cases}
     $\Pi=g\Pi_t$\\
     $D_t=ag\Pt_t+(1-g)$
   \end{cases}
  \end{equation*}
  $Y=MX+0$\\

  $
  \begin{vmatrix}
    D_t^*\\
    D_t
  \end{vmatrix}=
  \begin{vmatrix}
    g & 0\\
    ag(1-a)
  \end{vmatrix}+
  \begin{vmatrix}
    \Pi_t\\
    D_{t-1}
  \end{vmatrix}+
  \begin{vmatrix}
    0\\
    \varepsilon_t
  \end{vmatrix}
  $\\

  \newline\newline
  $R_t=a_1+b_12*Y_t+b_14*M_t$\\
  $Y_t=a_2+b_21*R_t+b_23*I_t+b_25*G_t$\\
  $I_t=a_3+b_31*R_t$\\

  \begin{vmatrix}
    -1 * b_{11} & 0\\
    b_{21} & 1 & -b_{23}\\
    -b_{31} & 0 & 1
  \end{vmatrix}+
  \begin{vmatrix}
    R_t\\
    Y_t\\
    T_t
  \end{vmatrix}+
  \begin{vmatrix}
    a_1 & b_{14} & 0\\
    a_1 & 0 & b_{23}\\
    a_3 & 0 & 0
  \end{vmatrix}+
  \begin{vmatrix}
    1\\
    M_1\\
    G_T
  \end{vmatrix}=
  \begin{vmatrix}
    0\\
    0\\
    0
  \end{vmatrix}\\

  A^{-1}=$\dfrac{1}{(b_{21}b_{12}-b_{31}b_{12}b_{23}-1)}$*
  \begin{bmatrix}
    -1 & -b_{12} & -b_{12}b_{23}\\
    -b_{21}-b_{b31} & b_{23}-1 & -b_{23}\\
    -b_{32} & -b_{21}b_{12} & b_{21}b_{12}-1
  \end{bmatrix}
  \newline

  2.3\\

  \begin{equation*}
   \begin{cases}
    $Y_t=C_t+I_t+Q_t+UX_t$ - тожество\\
    C_t=a(Y_t-T_{t-1})+b+U_t\\
    I_t=f+pR+\varepsilon_t, p<0\\
    L_t=dR+eY+Vt, d<0, e>0\\
    $R_t=\dfrac{M_t}{P_t}$ - тождество\\
    $NX_t=gE_t+W_{t}, g<0$\\
    $CE_t=hR+M_T$, h>0\\
    $CF_t=NX - тождество$
   \end{cases}
  \end{equation*}

  \begin{vmatrix}

  \end{vmatrix}
  \begin{vmatrix}
    Y_t\\
    C_t\\
    I_t\\
    L_t\\
    R_t\\
    NX_t\\
    CE_t\\
    CF_t
  \end{vmatrix}








%  Решение квадратного уравнения \(ax^2+bx+c=0\):
%  \begin{equation}
%    \label{eq:solv}
%    x_{1,2}=\frac{-b\pm\sqrt{b^2-4ac}}{2a}
%  \end{equation}
%
%  Можно сослаться на уравнение~\eqref{eq:solv}.

\end{document}